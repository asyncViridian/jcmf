The msa module implements input parsers and output routines for
multiple sequence alignment files.

\subsection{The msa API}

The module implements two objects. A \ccode{ESL\_MSA} holds a multiple
sequence alignment. A \ccode{ESL\_MSAFILE} is an alignment file,
opened for input. (No special object is needed for output of an
alignment file; a normal C \ccode{FILE} stream is used for output.)

The API consists of the following functions:

\vspace{1em}
\begin{tabular}{ll}\hline
   \multicolumn{2}{c}{\textbf{open/close of an msa file}}\\
\ccode{esl\_msafile\_Open()}                & Opens alignment file for input.\\
\ccode{esl\_msafile\_Close()}               & Closes an open alignment file.\\
   \multicolumn{2}{c}{\textbf{main msa i/o functions}}\\
\ccode{esl\_msa\_Read()}                    & Reads an MSA from open file.\\
\ccode{esl\_msa\_Write()}                   & Writes an MSA to open stream.\\
   \multicolumn{2}{c}{\textbf{MSA object maintenance}}\\
\ccode{esl\_msa\_Create()}                  & Creates a new MSA.\\
\ccode{esl\_msa\_Destroy()}                 & Destroys a created MSA.\\
\ccode{esl\_msa\_Expand()}                  & Expands allocation of an MSA.\\
   \multicolumn{2}{c}{\textbf{Format autodetermination}}\\
\ccode{esl\_msa\_GuessFileFormat()}         & Guesses an input file format.\\
   \multicolumn{2}{c}{\textbf{Alterations of an MSA}}\\
\ccode{esl\_msa\_NoGaps()}            & Removes columns containing any gaps.\\
\ccode{esl\_msa\_MinimGaps()}         & Removes columns containing only gaps.\\
\ccode{esl\_msa\_SymConvert()}        & A \ccode{tr//} op on the alignment.\\\hline
\end{tabular}

\subsection{Example of using the msa API}

Normally, you will only need to use five main functions from the API.

To read an alignment (or alignments) from a file, you open the file
with \ccode{esl\_msafile\_Open()} and read the alignment(s) with
\ccode{esl\_msa\_Read()}; when you're done with an alignment, free it
with \ccode{esl\_msa\_Destroy()}, and when you're done with the file,
close it with \ccode{esl\_msafile\_Close()}. 

If you want to output an alignment, you open a normal C \ccode{FILE}
stream, write the alignment(s) with \ccode{esl\_msa\_Write()}, and
close the stream with C's \ccode{fclose()}.

You may not need the other functions.  Normally, a new MSA is created
in the \ccode{esl\_msa\_Read()} call, not by calling
\ccode{esl\_msa\_Create()} directly, but the \ccode{\_Create()} and
\ccode{\_Expand()} functions are provided by the API in case you need
to write your own parser for a new format (and if so, you can follow
one of the existing parsers in \ccode{msa.c} as an example. Similarly,
the \ccode{esl\_msa\_GuessFileFormat()} function is normally called
only from within \ccode{esl\_msafile\_Open()} when you open a new
alignment file using format autodetection; but it's in the external
API in case you find it useful for some other purpose (like, checking
what Easel thinks a file's format is). And finally, the individual i/o
functions for specific formats are visible in the external API, but
you would normally just call the main \ccode{\_Read()} and
\ccode{\_Write()} wrapper functions instead.

Anyway, on to an example. The code bloats up a bit here because of
error handling. Lots of things can go wrong with input from a user's
file format, so the API tries to be careful about detecting different
kinds of errors. Here's code that reads one or more alignment in from
a file, and outputs them in Stockholm format:

\input{cexcerpts/msa_example}

Some things that are special about the API are worth noting here.

\begin{enumerate}
\item The format of the alignment file can either be automatically
      detected, or set by the caller when the file is opened.
      Autodetection is invoked when the caller passes a \ccode{fmt}
      code of \ccode{eslMSAFILE\_UNKNOWN}. Autodetection is a ``best
      effort'' guess, but it is not 100\% reliable - especially if the
      input file isn't an alignment file at all. Autodetection is a
      convenient default but the caller will usually want to provide a
      way for the user to specify the input file format and override
      autodetection, just in case.

\item If reading of an alignment fails because something is wrong with
      the file format, it's useful to give the user more information
      about what went wrong than just ``parse failed''. If
      \ccode{esl\_msa\_Read()} returns an \ccode{eslEFORMAT} error
      when trying to read an alignment from an open file \ccode{afp},
      the caller can use \ccode{afp->linenumber}, \ccode{afp->buf},
      and \ccode{afp->errbuf} to get the line number in the file that
      the error occurred, the text that was on that line, and a short
      error message about what was wrong with it, respectively.

\item The MSA module can be optionally \emph{augmented} by the keyhash
      module at compile-time. This improves the speed of parsing large 
      Stockholm format alignments.
\end{enumerate}

\subsection{Accessing alignment information in the MSA object}

The information in the \ccode{ESL\_MSA} object is meant to be accessed
directly, so you need to know what it contains. This object is defined
and documented in \ccode{msa.h}. It contains various information, as
follows:

\subsubsection{Important/mandatory information}

The following information is always available in an MSA.

\input{cexcerpts/msa_mandatory}

The alignment contains \ccode{nseq} sequences, each of which contains
\ccode{alen} characters.

\ccode{aseq[i]} is the i'th aligned sequence, numbered
\ccode{0..nseq-1}. \ccode{aseq[i][j]} is the j'th character in aligned
sequence i, numbered \ccode{0..alen-1}.

\ccode{sqname[i]} is the name of the i'th sequence.

\ccode{wgt[i]} is a non-negative real-valued weight for sequence
i. This defaults to 1.0 if the alignment file did not provide weight
data. You can determine whether weight data was parsed by checking
\ccode{flags \& MSA\_HASWGTS}.

\subsubsection{Optional information}

The following information is optional; it is usually only provided by
annotated Stockholm alignments (for instance, Pfam and Rfam database
alignments). 

Any pointer can be NULL if the information is unavailable. This is
true at any level; for instance, \ccode{ss} will be NULL if no
secondary structures are available for any sequence, and \ccode{ss[i]}
will be NULL if some secondary structures are available, but not for
sequence i. 

\input{cexcerpts/msa_optional}

These should be self-explanatory; but for more information, see the
Stockholm format documentation. Each of these fields corresponds to
Stockholm markup.

The \ccode{cutoff} array contains Pfam/Rfam curated score
cutoffs. They are indexed as follows:

\input{cexcerpts/msa_cutoffs}

\subsubsection{Unparsed information}

The MSA object may also contain additional ``unparsed'' information
from Stockholm files; that is, tags that are present but not
recognized by the MSA module. This information is stored so that it
may be regurgitated if the application needs to faithfully output the
entire alignment file, even the bits that it didn't understand. If you
need to access unparsed Stockholm tags, see the comments in
\ccode{msa.h}.


\subsection{Accepted formats}

Currently, the MSA module only parses Stockholm format. 

Stockholm format and other alignment formats are documented in a later
chapter.

\subsection{Reading from a pipe from compressed files}

The module can read compressed alignment files.  If the
\ccode{filename} passed to \ccode{esl\_msafile\_Open} ends in
\ccode{.gz}, the file is assumed to be compressed with gzip. Instead
of opening it normally, \ccode{esl\_msafile\_Open} opens it as a pipe
through \ccode{gunzip -dc}. Obviously this only works on a POSIX
system -- pipes have to work, specifically the \ccode{popen()} system
call -- and \ccode{gunzip} must be installed and in the PATH.

The module can also read from a standard input pipe. If the
\ccode{filename} passed to \ccode{esl\_msafile\_Open} is \ccode{-},
the alignment is read from \ccode{STDIN} rather than from a file.

Because of the way format autodetection works, you cannot use it when
reading from a pipe or compressed file. The application must know the
appropriate format and pass that code it calls
\ccode{esl\_msafile\_Open}.
